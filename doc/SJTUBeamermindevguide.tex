\documentclass{article}
\usepackage{array}
\usepackage[colorlinks,breaklinks=true]{hyperref}
\usepackage[numbers, sort&compress]{natbib}
\bibliographystyle{IEEEtran}
\def\tmin{\textsf{SJTUBeamer} \fbox{\textsc{min}}}
\title{Developer's Guide on\\ \tmin}
\author{Log Creative}
\date{\input{../VERSION}\today}
\begin{document}
    \maketitle
    \tableofcontents
    \clearpage
    \section{Preface}

    \tmin{} is a presentation template based on \textsf{beamer} package in \LaTeX{}, to fulfill the enthusiasm of those SJTU users to present their content nicely, benefiting from the technology of \TeX{} typesetting engine.
    
    This is a Developer's Guide on \tmin{} . The document is written in English because the operation in this guidance could be dangerous. Be careful when playing with those macros.

    \begin{quotation}
        \begin{center}
            \tmin{} --- the minimal work set of SJTU VI
        \end{center}
        \vspace*{1em}

        \fbox{\textsc{min}} - \emph{minimal}: \hfill minimal work set of SJTU VI.

        \fbox{\textsc{min}} - \emph{minimalism}:\hfill  designed in the style of minimalism.

        \fbox{\textsc{min}} - \emph{minimum}:\hfill  minimum shapes to show your content.
    \end{quotation}

    \section{Compliation}

    Most problems come from \LaTeX{} compilation. The required packages are in the following list.
    
    \begin{table}[h]
        \centering
        \begin{tabular}{>{\sffamily}c>{\sffamily}c>{\sffamily}c}
            \hline
            pgfplots & tikz & xcolor \\
            pgfplotstable & sansmath & tcolorbox \\
            ctex & biblatex & beamer \\
            \hline
        \end{tabular}
    \end{table}

    The detailed description is documented below.
    
    \subsection{MiK\TeX{}}

    All required packages will be automatically installed if you are using MiK\TeX{}\cite{miktex}. And if you want to use the \verb"latexmk" command, please install Perl\cite{perl} first. And the compilation command for \tmin{} is as follows:
    \begin{verbatim}
        latexmk -pdf main -interaction=nonstopmode
    \end{verbatim}

    \subsection{\TeX{} Live}
    Since some packages are not default installed in the full release of \TeX{} Live, you have to install the packages manually.

    On Ubuntu, you could install \verb"pgf" and \verb"xcolor" and other drawing packages through the following command\cite{beamerman}:

    \begin{verbatim}
        sudo apt install texlive-pictures
    \end{verbatim}

    To typeset Chinese characters, you would better use \verb"CJKutf8" package (in \tmin{}, set \verb"[cjk=true]"), since it is compatible with all platforms and multiple language support. Surround \verb"CJK" environment to make it work and remember to move all the Unicode characters in the permeable to the \verb"CJK" environment\cite{lsp3}:

    \begin{verbatim}
        \begin{document}
        \begin{CJK}{UTF8}{gbsn}
            \institute[]{}
            \title{}
            \subtitle{}
            \author{}
            \date{}
            % your content here ...
        \end{CJK}
        \end{document}
    \end{verbatim}

    However, if you stick into \verb"ctex",  you can install through \verb"tlmgr". If that works, then we call it a day.
    
    \begin{verbatim}
        sudo tlmgr install ctex
    \end{verbatim}
    
    Sometimes, you installed an old \TeX{} Live, and you have to upgrade the \verb"tlmgr" for the new version. And the process could be very buggy, since the following warning may be shown:

    \begin{verbatim}
        unexpected return value from verify_checksum: -5
    \end{verbatim}

    and to upgrade the \verb"tlmgr" is painful on Ubuntu. You should add the following content to \verb"/etc/profile/", which will add the newest path when the system is booting up\cite{upgrade}:
    \begin{verbatim}
        export PATH=/usr/local/texlive/2021/bin/x86_64-linux:
        /usr/local/texlive/:$PATH
    \end{verbatim}

    Reboot your computer if necessary. Then the compile system will be moved to the newer version of \TeX{} Live. Try to install the corresponding packages through the GUI interface of \verb"tlmgr":

    \begin{verbatim}
        sudo tlmgr update --self
        sudo tlmgr gui
    \end{verbatim}

    And if you encountered that
    \begin{verbatim}
        Critical Package ctex Error: CTeX fontset `fandol' is 
        unavailable in current(ctex) mode.
    \end{verbatim}

    You have to modify your compiling program from pdf\LaTeX{} to Xe\LaTeX{}  by adding the following magic command on the first line:

    \begin{verbatim}
        % !TeX TS-program = xelatex
    \end{verbatim}

    \subsection{Boost Up}

    However, it has been tested that the compilation on \tmin{} is slow. Since the complex patterns have to be rendered in vector shapes and the bibliography requires multiple times of compilation, the time could be wasted on some repetitive works.

    This scenario could be improved by enable \verb"[pattern=none]" option on \tmin{} and enable \verb"[draft]" option on beamer. The former one will disable all the pattern rendering, and the latter one will ignore all the TOC (table of contents) generating.

    The project has been implanted to Overleaf. 
    Here is the link \cite{overleaf}. And to make that works, the compilation on \TeX{} Live 2021 has to be implemented. And it is discovered that setting the document information outside the \verb|document| environment will cause a significantly longer compiling time, which may be caused by some improper settings in C\TeX{} package. The workaround of that is to follow the setup mentioned in \verb|CJK| settings: put that info into the body of document\cite{lsp3}.
    
    Currently, CI is available on Github Actions by compiling on Lua\LaTeX{}. \tmin{} uses \verb|xu-cheng/latex-action@v2| for the compilation docker \cite{lact} and relocates the compiling folder to \verb"src/". After compiling, output the PDF artifact. See \verb".github/workflows/main.yml" for details.

    At the same time, AutoBeamer\cite{ab} is making its own effort on generating beamer code automatically by some replacing strategies. You could preview your beamer code through conversion on Markdown or the article \LaTeX{} code.

    Furthermore, there is space for boosting up the beamer compilation time by making use of multi-core processors. Since it is a frame-based document, and the connection between each frame is loose (only some page numbers and citations need to be calculated), the multi-threaded compilation is possible for the \textsf{beamer} class. You can glimpse the multi-threaded processing for \LaTeX{} from the package \textsf{animate}. In fact, the author created some batch compiling work\cite{pgfedt} together with the \verb"-Parallel" parameter in PowerShell 7 to make full use of the concurrent computer architecture.
    
    \section{Modular Architecture}
    
    By the recommendation from \textsf{beamer} package\cite{beamerman}, \tmin{} uses the same modular architecture to build the template. Like it is in Java, to let the \textsf{beamer} template locate your theme, the style file has to be in the standard name.
    
    \begin{table}[h]
   		\begin{tabular}{>{\ttfamily}ll}
   			\bfseries .sty File & \bfseries Description \\
   			beamercolorthemeSJTUBeamermin.sty & Define global color schemes. \\
   			beamerfontthemeSJTUBeamermin.sty & Set the font format. \\
   			beamerinnerthemeSJTUBeamermin.sty & Specifies all parts inside a frame. \\
   			beamerouterthemeSJTUBeamermin.sty & The frame header and bottom bar. \\
   			beamerthemeSJTUBeamermin.sty & Entry point of the theme.
    	\end{tabular}
    \end{table}

	Notice that there are some dependencies (logo files) in the \verb|vi/|. Copying the \verb|vi| folder is necessary. Or you could define the location of the logo file by giving \verb"\logo{\includegraphics{logo.pdf}}".

    \begin{figure}[h]
        \framebox[\textwidth]{\ttfamily main.tex}
        
        \framebox[\textwidth]{\ttfamily beamerthemeSJTUBeamermin.sty}
        
        \framebox[0.25\textwidth]{\ttfamily colortheme.sty}\framebox[0.25\textwidth]{\ttfamily fonttheme.sty}\framebox[0.25\textwidth]{\ttfamily outertheme.sty}\framebox[0.25\textwidth]{\ttfamily innertheme.sty}

        \hfill\framebox[0.5\textwidth]{\ttfamily logo.pdf}
    \end{figure}

    \subsection{Theme}

    The main theme file \texttt{beamerthemeSJTUBeamermin.sty} is the entry point of the theme template. For users, after acquiring the \textsf{beamer} package, \verb"\usetheme" command will serve as the caller of the theme.
    \begin{verbatim}
        \documentclass{beamer}
        \mode<presentation>
        \usetheme{SJTUBeamermin}
    \end{verbatim}

    And this file will preprocess the option passed to the theme. Some options will be affected immediately, while others will get processed in the sub-style files.

    \begin{figure}[h]
        \fbox{\parbox{0.4\textwidth}{\texttt{theme.sty}\par
        \parbox{0.49\textwidth}{lang}\par
        \parbox{0.49\textwidth}{cjk}\par
        \parbox{0.49\textwidth}{gbt}\par
        \parbox{0.49\textwidth}{\emph{other settings}}
        }}
        \parbox{0.6\textwidth}{
            \fbox{\parbox{0.55\textwidth}{\texttt{colortheme.sty}\hfill color}}\par
            \fbox{\parbox{0.55\textwidth}{\texttt{fonttheme.sty}}}\par
            \fbox{\parbox{0.55\textwidth}{\texttt{outertheme.sty}\hfill pattern,navigation,lang}}\par
            \fbox{\parbox{0.55\textwidth}{\texttt{innertheme.sty}\hfill pattern,color,lang}}
        }
    \end{figure}

    \subsection{Color}

    The color style file \verb"beamercolorthemeSJTUBeamermin.sty" is the color setup of the template. Most color schemes are derived from the basic color of SJTU VI\cite{viman}. And to adapt the color definitions of \textsf{beamer}, the corresponding interface is mapped, see 17.2 in \cite{beamerman}.

    \begin{table}[h]
        \centering
        \begin{tabular}{>{\ttfamily}l>{\ttfamily}l>{\ttfamily}c>{\ttfamily}c}
             interface           & color=       & red       & blue \\
             palette primary     & cprimary     &\#004098  & \#9E1F36 \\       
             palette secondary   & csecondary   &\#298626  & \#F28101 \\     
             palette tertiary    & ctertiary    &\#004D4B  & \#FED201 \\      
             palette quanternary & cquanternary & \#FFFFFF  & \#000000 \\  
        \end{tabular}
    \end{table}

    As it is mapped to those beamer interfaces, to use the color, you have to declare the color struct first by
    \begin{verbatim}
        \usebeamercolor{palette primary}
        \color{palette primary.bg}
    \end{verbatim}
    or simply
    \begin{verbatim}
        \usebeamercolor[bg]{palette primary}
    \end{verbatim}

    However, there are scenarios where you cannot put temporary variables in some package options since it expands to \verb"\color{\color{mycolor}}". In this complex case, the redefinition of those standard colors is required. And that's the reason why \verb"innertheme.sty" gets \verb"color".

    \subsection{Font}

    The font style file \verb"beamerfontthemeSJTUBeamermin.sty" provides the font style of the beamer. In \tmin{}, serif math font is used by 
    \begin{verbatim}
        \usefonttheme{professionalfonts}
    \end{verbatim}
    which will tell \textsf{beamer} not to meddle with the specific font (in this case, math font) to the sans serif one. 
    
    It is especially useful if you don't want to create more compilation errors on many engines since some engine doesn't support sans serif math font. The workaround for that is to introduce the package below:
    \begin{verbatim}
        \RequirePackage[eulergreek]{sansmath}
    \end{verbatim}

    And \tmin{} does both.

    \subsection{Outer}

    The outer style file \verb"beamerouterthemeSJTUBeamermin.sty" contains the layout of frames. The recommended setup is as follows:
    \begin{table}[h]
        \centering
        \begin{tabular}{lc}
            \bfseries Components & \tmin \\
            head- and footline & $\bullet $ \\
            sidebars & \\
            logo & $\bullet$ \\
            frame title & $\bullet$\\
        \end{tabular}
    \end{table}

    \subsection{Inner}

    The inner style file \verb"beamerinnerthemeSJTUBeamermin.sty" will customize the main components. 
    \begin{table}[h]
        \centering
        \begin{tabular}{lc}
            \bfseries Components & \tmin \\
            Title and part pages & $\bullet $ \\
            Itemize & $\bullet $ \\
            Enumerate &  \\
            Description & \\
            Block & \\
            Theorem and proof & \\
            Figures and tables & $\bullet$ \\
            Footnotes & $\bullet$ \\
            Bibliography entries & \\
        \end{tabular}
    \end{table}

    Outer theme and inner theme are the core files for \tmin{}, which will be discussed in the following content.

    \bibliography{dev.bib}
\end{document}