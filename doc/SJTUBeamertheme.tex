\documentclass[a4paper,12pt]{article}
\usepackage{CJKutf8}
\usepackage{amsthm}
\usepackage{amsmath}
\usepackage{amssymb}
\usepackage{geometry}
\usepackage{graphicx}
\usepackage{array}
\usepackage{subfigure}
% \usepackage{gbt7714}
\usepackage{appendix}
\usepackage{listings}
\usepackage{xcolor}
\usepackage{tikz}
\usepackage{float}
\usepackage{tcolorbox}
\usepackage[ruled,vlined,commentsnumbered]{algorithm2e}
\usepackage{multicol}
\usepackage{pgfplots}
\pgfplotsset{compat=1.17}
\usepackage{indentfirst}
\usepackage{enumitem}
\usepackage[colorlinks,linkcolor=blue,anchorcolor=blue,citecolor=blue]{hyperref}
\hypersetup{unicode} % to display the unicode char in the bookmark correctly
\usepackage{bookmark} % should be introduced after package hyperref
\usepackage{listings}
\setlength{\parskip}{1ex}
\setenumerate[1]{itemsep=0pt,partopsep=0pt,parsep=\parskip,topsep=5pt}
\setitemize[1]{itemsep=0pt,partopsep=0pt,parsep=\parskip,topsep=5pt}
\setdescription{itemsep=0pt,partopsep=0pt,parsep=\parskip,topsep=5pt}
\geometry{left=2.0cm,right=2.0cm,top=2.0cm,bottom=2.0cm}
\renewcommand{\figurename}{图}
\renewcommand{\tablename}{表}
\renewcommand{\contentsname}{目录}
\tcbuselibrary{raster}
\tcbuselibrary{skins}
\tcbuselibrary{documentation}

\begin{document}
\begin{CJK}{UTF8}{song}
\title{\textsf{SJTUBeamer} 样式手册}
\author{Log Creative}
\date{\today}
\maketitle

\tableofcontents    % generate contents

\section{简介}

\textsf{SJTUBeamer} 样式为上海交通大学幻灯片模板的 \LaTeX{} 非官方实现版本,遵守\href{http://vi.sjtu.edu.cn/}{上海交通大学视觉形象识别系统}的相关规范,图标版权归上海交通大学所有,本项目仅供校内人员学习参考使用。

目前该样式仍然处于开发者预览版本( $\alpha$-\input{../VERSION}),仍在做适应性修改并适当添加图案。欢迎通过拉取请求对本模板提出修改建议。

\section{编译}

该模板的使用范例代码请见 \href{https://github.com/LogCreative/SJTUBeamer/blob/main/src/test.tex}{测试文件}。需要使用副本时请直接拷贝 \verb"src" 文件夹修改测试文件,目前版本不要轻易删除样式文件。

为了编译出目录,你需要编译两次,或者直接使用 \verb"latexmk" 编译。

使用 \href{https://github.com/LogCreative/SJTUBeamer/generate}{\texttt{Use This Template}} 按钮可以将本存储库附带的 Github Actions 自动编译设置同时拷贝,在提交修改 \verb"test.tex" 文件后会自动编译出 PDF 文件,置于 Github Actions 详情页面的 Artifacts 一栏。

\section{选项}

\begin{docKey*}[SJTUBeamer]{navigation}{=tools|subsections}{default \texttt{tools}}
    导航栏选项。默认为 \verb"tools",即工具栏与页码组合。设置为 \verb"subsections",将会产生子章节的跳转进度条。
\end{docKey*}

\begin{docKey*}[SJTUBeamer]{lang}{=cn|en}{default \texttt{cn}}
    语言选项,决定校徽是中文校徽,还是英文校徽。注意,启用 \texttt{en} 将会导致中文包不会被加载,因为使用英文徽标时,幻灯片内容推荐为英文。
\end{docKey*}

\begin{docKey*}[SJTUBeamer]{cjk}{=false|true}{default \texttt{false}}
    是否使用 \textsf{CJK} 宏包。如果开启此选项,请使用 \verb"\begin{CJK}{UTF8}{hei}" 和 \verb"\end{CJK}" 包裹 \verb"document" 环境中的内容。
\end{docKey*}

\begin{docKey*}[SJTUBeamer]{color}{=blue|red}{default \texttt{blue}}
    使用蓝色或红色的颜色主题。根据官方学术版模板 \href{http://vi.sjtu.edu.cn/index.php/articles/app/8}{\textbf{B2-02-02}} 的相关说明,极简版式适合场合较为严肃、不适合题图复杂图片出现的情况,如需使用,建议所有内页均使用极简版式,以达到简洁、清爽的效果。故默认情况下本模板使用蓝色主题。
\end{docKey*}

\begin{docKey*}[SJTUBeamer]{pattern}{=none|title|all}{default \texttt{title}}
    使用印记矩阵底纹。默认为 \verb"title",将会在标题页添加由 \href{http://vi.sjtu.edu.cn/index.php/articles/base/5}{\textbf{A5-03-01 辅助图形使用规范}} 规定的印记矩阵底纹,并采用了渐变底色的方式实现。如果选项为 \verb"none",就不会产生底纹;如果选项为 \verb"all",还会在每页的标题栏添加底纹,这可能会增加编译时间。
\end{docKey*}

\section{环境}

\begin{docEnvironment*}[doc parameter=\marg{language}\marg{title}, doclang/environment content=code]{codeblock}{}
    代码块。第一个参数为代码语言,第二个参数为代码块的标题。参数可以用 \verb"{}" 留空,但不要省略大括号。需要清理代码的缩进。注意代码块所在的幻灯片需要添加 \verb"fragile" 参数,也就是 \verb"\begin{frame}[fragile]"。
\end{docEnvironment*}

\begin{docEnvironment*}[doclang/environment content=code]{stampbox}{}
    图片印记边框盒。根据 \href{http://vi.sjtu.edu.cn/index.php/articles/base/5}{\textbf{A5-03-02 辅助图形使用规范}},可以使用辅助图形添加辅助边。这里采用了单一的印记边框线条,而没有使用填充式。该样式符合官方学术版模板 \href{http://vi.sjtu.edu.cn/index.php/articles/app/8}{\textbf{B2-02-02}} 的示例。\meta{code} 为需要包裹的内容。该边框盒不是强制使用的,主要为了丰富视觉效果与统一视觉形象。本模板通过程序自动生成边栏样条,不需要通过缩放方式对提供的样条进行适配。
\end{docEnvironment*}

\begin{docEnvironment*}[doclang/environment content=code]{axis}{}
    插入原生统计图。该模板对于使用 \textsf{PGFPlots} 生成的统计图进行了样式统一,主要改变了配色方案为当前主题,适当改变了线宽设定。由于统计图一般需要通过一些说明文字进行解释,所以在使用双栏版式时统计图的宽度将会被自动设定为栏宽。作者也制作了一个 \href{https://logcreative.github.io/PGFPlotsEdt/index.html?lang=cn}{\textbf{PGFPlots 统计绘图编辑器}},便于自动生成产生该类型统计图的 \LaTeX{} 代码,关于 \textsf{PGFPlots} 宏包的使用提示也可以在 \href{https://logcreative.github.io/LaTeXSparkle/}{像模像样\LaTeX} 的 \href{https://logcreative.github.io/LaTeXSparkle/src/art/chapter06.html}{第六节} 找到。
\end{docEnvironment*}

\section{命令}

\begin{docCommand*}[]{pgfplotstabletypeset}{\marg{file}}
    通过文件插入表格。这里默认表格文件为逗号分隔符格式(\verb".csv"),这种格式的表格可以通过 Excel 等外部软件导出而不必在 \LaTeX{} 内手动排版。本模板对于该类型插入的表格进行了样式统一,使用通用的三线表格并对表头使用主题颜色,一定程度上参考了 \href{https://vi.sjtu.edu.cn/index.php/articles/app/7}{\textbf{B1-16-01 档案袋}} 的表格设定。
\end{docCommand*}

以下命令不是 \textsf{Beamer} 的通用命令,迁移套用其他模板时请注释掉。

\begin{docCommand*}[]{makebottom}{}
    制作封底。本模板提供了 \href{https://vi.sjtu.edu.cn/index.php/articles/app/7}{\textbf{B1-20-01 文件封套(封面封底)}} 样式的封底。中文时,提示语为“谢谢”,英文时,提示语为``Thank You''。并且紧跟该幻灯片的作者与标题。
\end{docCommand*}

\begin{docCommand*}[]{stamparray}{\marg{size}\marg{starting point}\marg{ending point}}
    印记矩阵。用于在 \docAuxEnvironment*[]{tikzpicture} 环境内手动生成 \href{http://vi.sjtu.edu.cn/index.php/articles/base/5}{\textbf{A5-05 辅助底纹制作}} 规定的印记矩阵。其中,\meta{size} 用于定义印记大小,\meta{starting point},\meta{ending point} 用于定义矩形对角顶点,均为二维坐标 \verb"(x,y)"。 
\end{docCommand*}

\newpage
\section{样式}

校标使用\href{http://vi.sjtu.edu.cn/index.php/articles/base/4}{\textbf{校标组合的最小使用规范 A4-11}}的 18mm
 修正稿。安全空间满足\href{http://vi.sjtu.edu.cn/index.php/articles/base/4}{\textbf{校标组合的安全空间 A4-12}}的 $\frac{1}{5}$ 校标高度规定。

\subsection{活力红}

活力红版本,配色方案满足\href{http://vi.sjtu.edu.cn/index.php/articles/base/3}{\textbf{辅助色彩规范 A3-02-01}}。\hfill\texttt{color=red}

\begin{tcbraster}[raster columns=3,colframe=red,colback=white,
    colbacktitle=red!50!white,fonttitle=\small\bfseries\ttfamily,
    left=0pt,right=0pt,top=0pt,bottom=0pt,boxsep=0pt,boxrule=0.6pt,
    toptitle=1mm,bottomtitle=1mm,drop lifted shadow,center title,
    graphics pages={1,...,9}]
    \tcbincludepdf{pdf/red.pdf}
\end{tcbraster}

\newpage
\subsection{简约蓝}

简约蓝版本,配色方案满足\href{http://vi.sjtu.edu.cn/index.php/articles/base/3}{\textbf{辅助色彩规范 A3-02-02}}。\hfill \texttt{color=blue}

\begin{tcbraster}[raster columns=3,colframe=blue,colback=white,
    colbacktitle=blue!50!white,fonttitle=\small\bfseries\ttfamily,
    left=0pt,right=0pt,top=0pt,bottom=0pt,boxsep=0pt,boxrule=0.6pt,
    toptitle=1mm,bottomtitle=1mm,drop lifted shadow,center title,
    graphics pages={1,...,9}]
    \tcbincludepdf{pdf/blue.pdf}
\end{tcbraster}

\section{拓展阅读}
可以前往 \href{https://logcreative.github.io/LaTeXSparkle/}{像模像样\LaTeX} 的 \href{https://logcreative.github.io/LaTeXSparkle/src/art/chapter07.html}{第七节} 阅读有关 \verb"Beamer" 宏包使用的相关信息。

\scriptsize  

\newpage
Copyright 2021 Log Creative \& \LaTeX{} Sparkle Project

The copyright holder for the logo is SJTU. The template itself doesn't change the ownership of the related graphics in the guideline. The template is only available for non-commercial purposes.

This work may be distributed and/or modified under the
conditions of the \LaTeX{} Project Public License, either version 1.3 of this license or (at your option) any later version.

The latest version of this license is in
\begin{quotation}
    \href{http://www.latex-project.org/lppl.txt}{http://www.latex-project.org/lppl.txt}
\end{quotation}
and version 1.3 or later is part of all distributions of \LaTeX{}
version 2005/12/01 or later.

This work has the LPPL maintenance status `maintained'.

The Current Maintainer of this work is Log Creative.

% If this document has no reference,
% please use pdfLaTeX compilation once.
%\bibliography{ref}

\end{CJK}
\end{document}