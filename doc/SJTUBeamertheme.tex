\documentclass[a4paper,12pt]{article}
\usepackage{CJKutf8}
\usepackage{amsthm}
\usepackage{amsmath}
\usepackage{amssymb}
\usepackage{geometry}
\usepackage{graphicx}
\usepackage{array}
\usepackage{subfigure}
% \usepackage{gbt7714}
\usepackage{appendix}
\usepackage{listings}
\usepackage{xcolor}
\usepackage{tikz}
\usepackage{float}
\usepackage{tcolorbox}
\usepackage[ruled,vlined,commentsnumbered]{algorithm2e}
\usepackage{multicol}
\usepackage{pgfplots}
\pgfplotsset{compat=1.17}
\usepackage{indentfirst}
\usepackage{enumitem}
\usepackage[colorlinks,linkcolor=blue,anchorcolor=blue,citecolor=blue]{hyperref}
\hypersetup{unicode} % to display the unicode char in the bookmark correctly
\usepackage{bookmark} % should be introduced after package hyperref
\setlength{\parskip}{1ex}
\setenumerate[1]{itemsep=0pt,partopsep=0pt,parsep=\parskip,topsep=5pt}
\setitemize[1]{itemsep=0pt,partopsep=0pt,parsep=\parskip,topsep=5pt}
\setdescription{itemsep=0pt,partopsep=0pt,parsep=\parskip,topsep=5pt}
\geometry{left=2.0cm,right=2.0cm,top=2.0cm,bottom=2.0cm}
\renewcommand{\figurename}{图}
\renewcommand{\tablename}{表}
\renewcommand{\contentsname}{目录}
\tcbuselibrary{raster}
\tcbuselibrary{skins}
\tcbuselibrary{documentation}

\begin{document}
\begin{CJK}{UTF8}{song}
\title{\textsf{SJTUBeamer} 样式手册}
\author{Log Creative}
\date{\today}
\maketitle

\tableofcontents    % generate contents

\section{简介}

\textsf{SJTUBeamer} 样式为上海交通大学幻灯片模板的非官方实现版本,遵守\href{http://vi.sjtu.edu.cn/}{上海交通大学视觉形象识别系统}的相关规范,图标版权归上海交通大学所有,本项目仅供校内人员学习参考使用。

目前该样式仍然处于开发者预览版本($\alpha$-0.2),仍在做适应性修改并适当添加图案。欢迎通过拉取请求对本模板提出修改建议。

\section{编译}

该模板的使用范例代码请见 \href{https://github.com/LogCreative/SJTUBeamer/blob/main/src/test.tex}{测试文件}。需要使用副本时请直接拷贝 \verb"src" 文件夹修改测试文件,目前版本不要轻易删除样式文件。目前版本需要使用 \verb"\sjtutitle" 命令替代 \verb"\maketitle" 命令。

为了编译出目录,你需要编译两次,或者直接使用 \verb"latexmk" 编译。

使用 \href{https://github.com/LogCreative/SJTUBeamer/generate}{\texttt{Use This Template}} 按钮可以将本存储库附带的 Github Actions 自动编译设置同时拷贝,在提交修改 \verb"test.tex" 文件后会自动编译出 PDF 文件,置于 Github Actions 详情页面的 Artifacts 一栏。

\section{选项}

\begin{docKey}[SJTUBeamer]{navigation}{=tools|subsections}{default \texttt{tools}}
    导航栏选项。默认为 \verb"tools",即工具栏与页码组合。设置为 \verb"subsections",将会产生子章节的跳转进度条。
\end{docKey}

\begin{docKey}[SJTUBeamer]{lang}{=cn|en}{default \texttt{cn}}
    语言选项,决定校徽是中文校徽,还是英文校徽。注意,启用 \texttt{en} 将会导致中文包不会被加载,因为使用英文徽标时,幻灯片内容推荐为英文。
\end{docKey}

\begin{docKey}[SJTUBeamer]{cjk}{=false|true}{default \texttt{false}}
    是否使用 \textsf{CJK} 宏包。如果开启此选项,请使用 \verb"\begin{CJK}{UTF8}{hei}" 和 \verb"\end{CJK}" 包裹 \verb"document" 环境中的内容。
\end{docKey}

\section{样式}

校标使用\href{http://vi.sjtu.edu.cn/index.php/articles/base/4}{\textbf{校标组合的最小使用规范 A4-11}}的 18mm
 修正稿。安全空间满足\href{http://vi.sjtu.edu.cn/index.php/articles/base/4}{\textbf{校标组合的安全空间 A4-12}}的 $\frac{1}{5}$ 校标高度规定。

\subsection{简约蓝}

目前仅仅实现了简约蓝版本,配色方案满足\href{http://vi.sjtu.edu.cn/index.php/articles/base/3}{\textbf{辅助色彩规范 A3-02-02}}。

\begin{tcbraster}[raster columns=2,colframe=blue,colback=white,
    colbacktitle=blue!50!white,fonttitle=\small\bfseries\ttfamily,
    left=0pt,right=0pt,top=0pt,bottom=0pt,boxsep=0pt,boxrule=0.6pt,
    toptitle=1mm,bottomtitle=1mm,drop lifted shadow,center title,
    graphics pages={1,...,5}]
    \tcbincludepdf{../src/test.pdf}
\end{tcbraster}

\section{拓展阅读}
可以前往 \href{https://logcreative.github.io/LaTeXSparkle/}{像模像样\LaTeX} 的 \href{https://logcreative.github.io/LaTeXSparkle/src/art/chapter07.html}{第七节} 阅读有关 \verb"Beamer" 宏包使用的相关信息。

\scriptsize  

\newpage
Copyright 2021 Log Creative \& \LaTeX{} Sparkle Project

This work may be distributed and/or modified under the
conditions of the \LaTeX{} Project Public License, either version 1.3 of this license or (at your option) any later version.

The latest version of this license is in
\begin{quotation}
    \href{http://www.latex-project.org/lppl.txt}{http://www.latex-project.org/lppl.txt}
\end{quotation}
and version 1.3 or later is part of all distributions of \LaTeX{}
version 2005/12/01 or later.

This work has the LPPL maintenance status `maintained'.

The Current Maintainer of this work is Log Creative.

% If this document has no reference,
% please use pdfLaTeX compilation once.
%\bibliography{ref}

\end{CJK}
\end{document}